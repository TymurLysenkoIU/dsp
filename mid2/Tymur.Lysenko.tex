\documentclass[a4paper]{article}

\usepackage[utf8]{inputenc}
\usepackage[margin=2cm]{geometry}
\usepackage[hidelinks]{hyperref}
\usepackage{textcomp,fullpage,enumitem,amssymb,amsmath,xcolor,cancel,gensymb,graphicx,indentfirst,xargs,mathtools,mathdots}
\usepackage[T1]{fontenc}
\usepackage[affil-it]{authblk}
\usepackage[english]{babel}
\usepackage[amsmath]{ntheorem}

%\begin{align*}…\end{align*} if you want to fit an equation. 
%FOR PICTURES: include graphicsx 
%\includegraphics[scale=x]{name}
%double space and write caption in the center class

% Document settings

\setlength{\parskip}{1em}

% Theorems

\theoremstyle{break}
\theorembodyfont{\normalfont}
\newtheorem{definition}{Definition}[section]

\theoremstyle{break}
\theoremheaderfont{\itshape\hspace{-\theoremindent}}
\theoremindent10pt
\newtheorem{property}{Property}[section]

% Custom operations

\newcommandx{\idx}[2][2=n]{\ensuremath{\left(#1\right)_{#2}} }

% \newcommandx{\innerp}[2][1=\_, 2=\_]{\ensuremath{\langle#1, #2\rangle} }
% \newcommandx{\taskop}[2][1=\_, 2=\_]{\ensuremath{(#1 | #2)} }

% \newcommandx{\W}[2][1=kn, 2=m]{\ensuremath{W_{#2}^{#1}} }
% \newcommandx{\Wval}[2][1=kn, 2=m]{\ensuremath{e^{-\frac{2 \cdot j \cdot \pi \cdot #1 }{#2}}} }
% \newcommandx{\Wvalcoef}[2][1=2kn, 2=m]{\ensuremath{e^{-\frac{#1 \cdot j \cdot \pi }{#2}}} }
% \newcommandx{\eulersvalue}[2][1=x, 2=+]{\ensuremath{\cos(#1) #2 j\sin(#1)} }

% \DeclarePairedDelimiter\abs{\lvert}{\rvert}


% Title

\title{Digital Signal Processing. Second Midterm}
\author{Tymur Lysenko}
\affil{BS18-02SE, Innopolis University}
\date{Date of Birth: 05.07.1999}


% Document

\begin{document}

\maketitle

\section*{Variant}

\begin{flalign*}
  &day = 5& \\
  &month = 7& \\
  &year = 1999& \\
\end{flalign*}

\section{Task 1}

\subsection*{Problem statement}

\newcommandx{\taskonesignal}[2][1=day, 2=month]{\ensuremath{\left(\frac{\sqrt{#1}}{\sqrt{#1 + #2}}, \frac{\sqrt{#2}}{\sqrt{#1 + #2}}\right)} }
\newcommandx{\taskonesignalvalue}[0][]{\ensuremath{\left(\frac{\sqrt{5}}{\sqrt{12}}, \frac{\sqrt{7}}{\sqrt{12}}\right)} }

Starting with the definitions, compute (according to your actual variant) cross-correlation of the box signal of length 3 with a finite signal \taskonesignal = \taskonesignalvalue.

\subsection*{Solution}

Start with the definitions.

\newcommandx{\boxsignal}[1][1=n]{\ensuremath{box^{#1}} }

\newcommandx{\boxsignalvalue}[1][1=n]{\ensuremath{\left( \cdots w_{-1} = 0 \quad \boldsymbol{w_0 = 1} \quad \cdots \quad w_{#1 - 1} = 1 \quad w_{#1} = 0 \cdots \right)} }

\newcommandx{\boxsignalvaluethree}[0][]{\ensuremath{\left( \cdots \; 0 \quad 0 \quad \boldsymbol{1} \quad 1 \quad 1 \quad 0 \quad 0 \; \cdots \right)} }

\begin{definition}[Box signal]
  Box signal of length $n > 0$ is a two-side infinite sequence, s. t. all elements are 0, except those starting from the 0-th element until $n - 1$-st element:

  \begin{equation*}
    \boxsignal = \boxsignalvalue
  \end{equation*}
\end{definition}

\begin{definition}[Cross-correlation ($\star$)]
  Cross-correlation of 2 complex infinite signals $x$ and $y$, denoted as $\star$ (star), is an infinite signal c, $n$-th element of which is calculated as follows:

  \begin{equation*}
    c_n = (x \star y)_n = \sum_{k = -\infty}^{k = +\infty} x_k^* y_{k + n},
  \end{equation*}

  where $^*$ is a complex conjugate.
\end{definition}

Now the cross-correlation can be computed for the task.

\begin{equation*}
  \begin{split}
    c = \boxsignal[3] \star \taskonesignalvalue = \boxsignalvaluethree \star \taskonesignalvalue
  \end{split}
\end{equation*}

For $n$ outside of range $[-2, 1]$, $c_n = 0$, since the \taskonesignalvalue will be multiplied with 0-s element-wise to calculate the cross-correlation, hence the sum of the products will be 0. It is only left to calculate non-zero elements of the cross-correlation:

\begin{alignat*}{2}
  \idx{c}[-2]
    &= \idx{\boxsignal[3]}[2]^* \cdot \taskonesignalvalue_0 &+ &\idx{\boxsignal[3]}[3]^* \cdot \taskonesignalvalue_1 \\
    &= 1 \cdot \frac{\sqrt{5}}{\sqrt{12}} &+ &0 \cdot \frac{\sqrt{7}}{\sqrt{12}} \\
    &= \frac{\sqrt{5}}{\sqrt{12}} && \\
  \\
  \idx{c}[-1]
    &= \idx{\boxsignal[3]}[1]^* \cdot \taskonesignalvalue_0 &+ &\idx{\boxsignal[3]}[2]^* \cdot \taskonesignalvalue_1 \\
    &= 1 \cdot \frac{\sqrt{5}}{\sqrt{12}} &+ &1 \cdot \frac{\sqrt{7}}{\sqrt{12}} \\
    &= \frac{\sqrt{5} + \sqrt{7}}{\sqrt{12}} && \\
    \\
    \idx{c}[0]
      &= \idx{\boxsignal[3]}[0]^* \cdot \taskonesignalvalue_0 &+ &\idx{\boxsignal[3]}[1]^* \cdot \taskonesignalvalue_1 \\
      &= 1 \cdot \frac{\sqrt{5}}{\sqrt{12}} &+ &1 \cdot \frac{\sqrt{7}}{\sqrt{12}} \\
      &= \frac{\sqrt{5} + \sqrt{7}}{\sqrt{12}} && \\
    \\
    \idx{c}[1]
      &= \idx{\boxsignal[3]}[-1]^* \cdot \taskonesignalvalue_0 &+ &\idx{\boxsignal[3]}[0]^* \cdot \taskonesignalvalue_1 \\
      &= 0 \cdot \frac{\sqrt{5}}{\sqrt{12}} &+ &1 \cdot \frac{\sqrt{7}}{\sqrt{12}} \\
      &= \frac{\sqrt{7}}{\sqrt{12}} && \\
\end{alignat*}

\subsection*{Conclusion}

\begin{equation*}
  \begin{split}
    c = \boxsignal[3] \star \taskonesignalvalue = (\cdots \; 0 \quad \frac{\sqrt{5}}{\sqrt{12}} \quad \frac{\sqrt{5} + \sqrt{7}}{\sqrt{12}} \quad \boldsymbol{\frac{\sqrt{5} + \sqrt{7}}{\sqrt{12}}} \quad \frac{\sqrt{7}}{\sqrt{12}} \quad 0 \; \cdots)
  \end{split}
\end{equation*}

\section{Task 2}

\subsection*{Problem statement}

\newcommandx{\tasktwosystem}[2][1=(day), 2=(month)]{\ensuremath{#1 A - #2 D} }
\newcommandx{\tasktwosystemforvariant}[0][]{\tasktwosystem[5][7]}

Let $A$ and $D$ be the \textit{advance} and \textit{delay} operators. Starting with definitions, study linearity of the system
(according to your actual variant) $T = \tasktwosystem = \tasktwosystemforvariant$, whether the system is causal, memoryless, BIBO-stable, LSI, sketch (draw) its matrix (if possible).

\subsection*{Solution}

\subsubsection*{Definitions}

Start with the definitions.

\begin{definition}[Memoryless linear system]
  A linear system $T$ is \textit{memoryless} if its matrix is diagonal.

  \begin{equation*}
    T = \begin{pmatrix*}
          \ddots & \vdots & \vdots & \vdots & \iddots \\
          \cdots & a^{-1}_{-1} & 0 & 0 & \cdots \\
          \cdots & 0 & \boldsymbol{a^{0}_{0}} & 0 & \cdots \\
          \cdots & 0 & 0 & a^{1}_{1} & \cdots \\
          \iddots & \vdots & \vdots & \vdots & \ddots
        \end{pmatrix*}
  \end{equation*}
\end{definition}

\begin{definition}[Causal linear system]
  A linear system $T$ is \textit{causal} if its matrix is lower triangular.

  \begin{equation*}
    T = \begin{pmatrix*}
          \ddots & \vdots & \vdots & \vdots & \iddots \\
          \cdots & a^{-1}_{-1} & 0 & 0 & \cdots \\
          \cdots & a^{0}_{-1} & \boldsymbol{a^{0}_{0}} & 0 & \cdots \\
          \cdots & a^{1}_{-1} & a^{1}_{0} & a^{1}_{1} & \cdots \\
          \iddots & \vdots & \vdots & \vdots & \ddots
        \end{pmatrix*}
  \end{equation*}
\end{definition}

\begin{definition}[LSI system (Linear Shift Invariant)]
  A linear system $T$ is \textit{linear shift invariant} if its matrix consists only of diagonals of some constants.

  \begin{equation*}
    T = \begin{pmatrix*}
          \ddots  & \vdots & \vdots             & \vdots & \iddots \\
          \cdots  & a_{0}  & a_{-1}             & a_{-2} & \cdots \\
          \cdots  & a_{1}  & \boldsymbol{a_{0}} & a_{-1} & \cdots \\
          \cdots  & a_{2}  & a_{1}              & a_{0}  & \cdots \\
          \iddots & \vdots & \vdots             & \vdots & \ddots
        \end{pmatrix*}
  \end{equation*}
\end{definition}

\begin{definition}[Impulse response of a LSI system]
  \textit{Impulse response $a$} of LSI system $T$ is the sequence obtained by applying $T$ to the Kronecker delta sequence $\delta = \left(\cdots \quad 0 \quad \boldsymbol{1} \quad 0 \quad \cdots\right)$.

  \begin{equation*}
    a = T \delta = \left( \cdots \quad a_{-1} \quad \boldsymbol{a_{0}} \quad a_{1} \quad \cdots \right)
  \end{equation*}
\end{definition}

\begin{definition}[BIBO-stable LSI system (Bounded-Input, Bounded-Output)]
  An LSI system $T$ is \textit{bounded-input, bounded-output stable} if its impulse response is absolutely summable.
\end{definition}

\begin{definition}[Advance operator]
  A linear system $A$ is called an \textit{advance} operator if its matrix has a diagonal of 1-s above the main diagonal and 0-es everywhere else.

  The shift degree is determined by the index distance from the central element until the first 1 either horizontally or vertically (the distance is the same) and is indicated in the super index of the operator, e. g. advance by 1 operator is written as

  \begin{equation*}
    A^{1} = \begin{pmatrix*}
          \ddots  & \vdots & \vdots         & \vdots & \iddots \\
          \cdots  & 0      & 1              & 0      & \cdots \\
          \cdots  & 0      & \boldsymbol{0} & 1      & \cdots \\
          \cdots  & 0      & 0              & 0      & \cdots \\
          \iddots & \vdots & \vdots         & \vdots & \ddots
        \end{pmatrix*}
  \end{equation*}

  When an operator is written without the super index it means, that the shift degree is not relevant and the statement applies to all positive shift degrees.
\end{definition}

\begin{definition}[Delay operator]
  A linear system $D$ is called a \textit{delay} operator if its matrix has a diagonal of 1-s below the main diagonal and 0-es everywhere else.

  The shift degree is determined by the index distance from the central element until the first 1 either horizontally or vertically (the distance is the same) and is indicated in the super index of the operator, e. g. delay by 1 operator is written as

  \begin{equation*}
    D^{1} = \begin{pmatrix*}
          \ddots  & \vdots & \vdots         & \vdots & \iddots \\
          \cdots  & 0      & 0              & 0      & \cdots \\
          \cdots  & 1      & \boldsymbol{0} & 0      & \cdots \\
          \cdots  & 0      & 1              & 0      & \cdots \\
          \iddots & \vdots & \vdots         & \vdots & \ddots
        \end{pmatrix*}
  \end{equation*}

  When an operator is written without the super index it means, that the shift degree is not relevant and the statement applies to all positive shift degrees.
\end{definition}

\subsubsection*{Linearity} \label{sssec:task-two-linearity}

Since, by definition, \textit{advance} and \textit{delay} operators are linear operators, the linear combination $T = \tasktwosystemforvariant$ \textbf{is also a linear operator}.

The system $T$ is of the form:

\begin{itemize}
  \item a diagonal above the main diagonal consisting only of 5-s
  \item a diagonal below the main diagonal consisting only of 7-s
  \item all other elements are 0-s
\end{itemize}

\subsubsection*{Memorylessness}

Because of the above, it is clear that the $T$ is not a diagonal matrix, so the system is \textbf{not memoryless}.

\subsubsection*{Causality}

$T$ is not represented as a lower triangular matrix, because there are elements above the main diagonal, hence the system is \textbf{not causal}.

\subsubsection*{Linear shift invariance}

Nevertheless, the system \textbf{is LSI}, because it has 2 diagonals of repeating values and 0-es elsewhere.

\subsubsection*{BIBO-stability}

Impulse response of the system is a sequence of all zeroes and 2 numbers 5 and -7, absolute sum of which is 12. Hence, the sequence is absolutely summable and hence \textbf{is BIBO stable}.

\subsection*{Conclusion}

The system has approximately (for more details see \nameref{sssec:task-two-linearity}) the following representation:

\begin{equation*}
  T = \tasktwosystemforvariant =
    \begin{pmatrix*}
      \ddots  & \vdots & \vdots  & \vdots  & \vdots & \iddots \\
      \cdots  & 0      & \cdots  & 5       & 0      & \cdots \\
      \cdots  & \vdots & 0       & \iddots & 5      & \cdots \\
      \cdots  & -7     & \iddots & 0       & \vdots & \cdots \\
      \cdots  & 0      & -7      & \cdots  & 0      & \cdots \\
      \iddots & \vdots & \vdots  & \vdots  & \vdots & \ddots
    \end{pmatrix*}
\end{equation*}

The system is:

\begin{itemize}
  \item \textbf{linear}
  \item \textbf{not memoryless}
  \item \textbf{not causal}
  \item \textbf{LSI}
  \item \textbf{BIBO stable}
\end{itemize}

\section{Task 3}

\subsection*{Problem statement}

\newcommandx{\taskthreex}[2][1=-day - month, 2=day-month]{\ensuremath{\left( #1 \quad \boldsymbol{0} \quad #2 \right)}}

\newcommandx{\taskthreexvalue}[0][]{\taskthreex[-12][-2]}

\newcommandx{\taskthreey}[0][]{\ensuremath{\left( 1 \quad 0 \quad \boldsymbol{1} \quad 0 \quad 1 \right)}}

Starting with the definitions, compute (according to the definition) convolution of the following two signals (according to your actual variant)

\begin{itemize}
  \item $x = \taskthreex = \taskthreexvalue$
  \item $y = \taskthreey$
\end{itemize}

Then compute in an efficient way convolution of the following pairs of signals:

\begin{itemize}
  \item $x$ and $0.5 \cdot D \cdot y$
  \item $x$ and $\left( 0.5 \cdot D \cdot y + y \right)$
  \item $A_{2} x$ and $D_{2} y$
\end{itemize}

(Here $A$ and $D$ are the advance and delay operators. Do not forget to explain the method of your computations and why it is efficient!)

\subsection*{Solution}

\subsubsection*{Definitions}

Start with the definitions and theorems.

\begin{definition}[(Linear) convolution ($\ast$)]
  (Linear) convolution of 2 complex-valued signals $x$ and $y$, denoted as $x \ast y$ is a complex-valued signal, $n$-th element of which is calculated as follows:

  \begin{equation*}
    \left(x \ast y \right)_{n} = \sum_{k = -\infty} ^{k = +\infty} x_{n - k} y_k = \sum_{k = -\infty} ^{k = +\infty} x_k y_{n - k}
  \end{equation*}
\end{definition}

\subsubsection*{Convolution properties}

\begin{property}[Commutativity]
  \begin{equation*}
    x \ast y = y \ast x
  \end{equation*}
\end{property}

\begin{property}[Linear and shift invariance]
  Convolution is linear and shift invariant in both arguments.

  \begin{equation*}
    (a x + b y) \ast z = a (x \ast z) + b (y \ast z)
  \end{equation*}
  \begin{equation*}
    x \ast (a y + b z) = a (x \ast y) + b (x \ast z)
  \end{equation*}
\end{property}

\subsubsection*{Calculations}

\begin{equation*}
  x \ast y = \taskthreexvalue \ast \taskthreey
\end{equation*}

The lower index of the convolution output signal is -3 and the highest is 3, because otherwise there will not be elements to sum.

\begin{alignat*}{3}
  (x \ast y)_{-3} &= \taskthreexvalue_{-1} \ast \taskthreey_{-2} &&= (-12 \cdot 1) &&= -12 \\
  (x \ast y)_{-2} &= \taskthreexvalue_{-1} \ast \taskthreey_{-1} &&&&\\
    &+ \taskthreexvalue_{0} \ast \taskthreey_{-2} &&= (-12 \cdot 0) + (0 \cdot 1) &&= 0 \\
  (x \ast y)_{-1} &= \taskthreexvalue_{-1} \ast \taskthreey_{0} + \\ 
    &+ \taskthreexvalue_{0} \ast \taskthreey_{-1} + &&&& \\
    &+ \taskthreexvalue_{1} \ast \taskthreey_{-2} &&= (-12 \cdot 1) + (0 \cdot 0) + (-2 \cdot 1) &&= -14 \\
  (x \ast y)_{0} &= \taskthreexvalue_{-1} \ast \taskthreey_{1} + &&&&\\
    &+ \taskthreexvalue_{0} \ast \taskthreey_{0} &&&&\\
    &+ \taskthreexvalue_{1} \ast \taskthreey_{-1} &&= (-12 \cdot 0) + (0 \cdot 1) + (-2 \cdot 0) &&= 0 \\
  % TODO: 1
  % TODO: 2
  % TODO: 3
\end{alignat*}

\end{document}

\documentclass[a4paper]{article}

\usepackage[utf8]{inputenc}
\usepackage[margin=2cm]{geometry}
\usepackage[hidelinks]{hyperref}
\usepackage{textcomp,fullpage,enumitem,amssymb,amsmath,xcolor,cancel,gensymb,graphicx,indentfirst,xargs,mathtools}
\usepackage[T1]{fontenc}
\usepackage[affil-it]{authblk}
\usepackage[english]{babel}
\usepackage[amsmath]{ntheorem}

%\begin{align*}…\end{align*} if you want to fit an equation. 
%FOR PICTURES: include graphicsx 
%\includegraphics[scale=x]{name}
%double space and write caption in the center class

% Document settings

\setlength{\parskip}{1em}

% Theorems

\theoremstyle{break}
\theorembodyfont{\normalfont}
\newtheorem{definition}{Definition}[section]

\theoremstyle{break}
\theoremheaderfont{\itshape\hspace{-\theoremindent}}
\theoremindent10pt
\newtheorem{property}{Property}[section]

% Custom operations

\newcommandx{\idx}[2][2=n]{\ensuremath{\left(#1\right)_{#2}} }

% \newcommandx{\innerp}[2][1=\_, 2=\_]{\ensuremath{\langle#1, #2\rangle} }
% \newcommandx{\taskop}[2][1=\_, 2=\_]{\ensuremath{(#1 | #2)} }

% \newcommandx{\W}[2][1=kn, 2=m]{\ensuremath{W_{#2}^{#1}} }
% \newcommandx{\Wval}[2][1=kn, 2=m]{\ensuremath{e^{-\frac{2 \cdot j \cdot \pi \cdot #1 }{#2}}} }
% \newcommandx{\Wvalcoef}[2][1=2kn, 2=m]{\ensuremath{e^{-\frac{#1 \cdot j \cdot \pi }{#2}}} }
% \newcommandx{\eulersvalue}[2][1=x, 2=+]{\ensuremath{\cos(#1) #2 j\sin(#1)} }

% \DeclarePairedDelimiter\abs{\lvert}{\rvert}


% Title

\title{Digital Signal Processing. Second Midterm}
\author{Tymur Lysenko}
\affil{BS18-02SE, Innopolis University}
\date{Date of Birth: 05.07.1999}


% Document

\begin{document}

\maketitle

\section*{Variant}

\begin{flalign*}
  &day = 5& \\
  &month = 7& \\
  &year = 1999& \\
\end{flalign*}

\section{Task 1}

\subsection*{Problem statement}

\newcommandx{\taskonesignal}[2][1=day, 2=month]{\ensuremath{\left(\frac{\sqrt{#1}}{\sqrt{#1 + #2}}, \frac{\sqrt{#2}}{\sqrt{#1 + #2}}\right)} }
\newcommandx{\taskonesignalvalue}[0][]{\ensuremath{\left(\frac{\sqrt{5}}{\sqrt{12}}, \frac{\sqrt{7}}{\sqrt{12}}\right)} }

Starting with the definitions, compute (according to your actual variant) cross-correlation of the box signal of length 3 with a finite signal \taskonesignal.

\subsection*{Solution}

Start with the definitions.

\newcommandx{\boxsignal}[1][1=n]{\ensuremath{box^{#1}} }

\newcommandx{\boxsignalvalue}[1][1=n]{\ensuremath{\left( \cdots w_{-1} = 0 \quad \boldsymbol{w_0 = 1} \quad \cdots \quad w_{#1 - 1} = 1 \quad w_{#1} = 0 \cdots \right)} }

\newcommandx{\boxsignalvaluethree}[0][]{\ensuremath{\left( \cdots 0 \quad 0 \quad \boldsymbol{1} \quad 1 \quad 1 \quad 0 \quad 0 \cdots \right)} }

\begin{definition}[Box signal]
  Box signal of length $n > 0$ is a two-side infinite sequence, s. t. all elements are 0, except those starting from the 0-th element until $n - 1$-st element:

  \begin{equation*}
    \boxsignal = \boxsignalvalue
  \end{equation*}
\end{definition}

\begin{definition}[Cross-correlation ($\star$)]
  Cross-correlation of 2 complex infinite signals $x$ and $y$, denoted as $\star$ (star), is an infinite signal c, $n$-th element of which is calculated as follows:

  \begin{equation*}
    c_n = (x \star y)_n = \sum_{k = -\infty}^{k = +\infty} x_k^* y_{k + n},
  \end{equation*}

  where $^*$ is a complex conjugate.
\end{definition}

Now the cross-correlation can be computed for the task.

\begin{equation*}
  \begin{split}
    c = \boxsignal[3] \star \taskonesignalvalue = \boxsignalvaluethree \star \taskonesignalvalue
  \end{split}
\end{equation*}

For $n$ outside of range $[-2, 1]$, $c_n = 0$, since the \taskonesignalvalue will be multiplied with 0-s element-wise to calculate the cross-correlation, hence the sum of the products will be 0. It is only left to calculate non-zero elements of the cross-correlation:

\begin{alignat*}{2}
  \idx{c}[-2]
    &= \idx{\boxsignal[3]}[2]^* \cdot \taskonesignalvalue_0 &+ &\idx{\boxsignal[3]}[3]^* \cdot \taskonesignalvalue_1 \\
    &= 1 \cdot \frac{\sqrt{5}}{\sqrt{12}} &+ &0 \cdot \frac{\sqrt{7}}{\sqrt{12}} \\
    &= \frac{\sqrt{5}}{\sqrt{12}} && \\
  \\
  \idx{c}[-1]
    &= \idx{\boxsignal[3]}[1]^* \cdot \taskonesignalvalue_0 &+ &\idx{\boxsignal[3]}[2]^* \cdot \taskonesignalvalue_1 \\
    &= 1 \cdot \frac{\sqrt{5}}{\sqrt{12}} &+ &1 \cdot \frac{\sqrt{7}}{\sqrt{12}} \\
    &= \frac{\sqrt{5} + \sqrt{7}}{\sqrt{12}} && \\
    \\
    \idx{c}[0]
      &= \idx{\boxsignal[3]}[0]^* \cdot \taskonesignalvalue_0 &+ &\idx{\boxsignal[3]}[1]^* \cdot \taskonesignalvalue_1 \\
      &= 1 \cdot \frac{\sqrt{5}}{\sqrt{12}} &+ &1 \cdot \frac{\sqrt{7}}{\sqrt{12}} \\
      &= \frac{\sqrt{5} + \sqrt{7}}{\sqrt{12}} && \\
    \\
    \idx{c}[1]
      &= \idx{\boxsignal[3]}[-1]^* \cdot \taskonesignalvalue_0 &+ &\idx{\boxsignal[3]}[0]^* \cdot \taskonesignalvalue_1 \\
      &= 0 \cdot \frac{\sqrt{5}}{\sqrt{12}} &+ &1 \cdot \frac{\sqrt{7}}{\sqrt{12}} \\
      &= \frac{\sqrt{7}}{\sqrt{12}} && \\
\end{alignat*}

\subsection*{Conclusion}

\begin{equation*}
  \begin{split}
    c = \boxsignal[3] \star \taskonesignalvalue = (\cdots 0 \quad \frac{\sqrt{5}}{\sqrt{12}} \quad \frac{\sqrt{5} + \sqrt{7}}{\sqrt{12}} \quad \boldsymbol{\frac{\sqrt{5} + \sqrt{7}}{\sqrt{12}}} \quad \frac{\sqrt{7}}{\sqrt{12}} \quad 0 \cdots)
  \end{split}
\end{equation*}

\end{document}

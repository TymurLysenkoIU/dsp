\documentclass[a4paper]{article}

\usepackage[utf8]{inputenc}
\usepackage[margin=2cm]{geometry}
\usepackage[hidelinks]{hyperref}
\usepackage{textcomp,fullpage,enumitem,amssymb,amsmath,xcolor,cancel,gensymb,graphicx,indentfirst,xargs,mathtools,mathdots}
\usepackage[T1]{fontenc}
\usepackage[affil-it]{authblk}
\usepackage[english]{babel}
\usepackage[amsmath]{ntheorem}

%\begin{align*}…\end{align*} if you want to fit an equation. 
%FOR PICTURES: include graphicsx 
%\includegraphics[scale=x]{name}
%double space and write caption in the center class

% Document settings

\setlength{\parskip}{1em}

% Theorems

\theoremstyle{break}
\theorembodyfont{\normalfont}
\newtheorem{definition}{Definition}[section]

\theoremstyle{break}
\theoremheaderfont{\itshape\hspace{-\theoremindent}}
\theoremindent10pt
\newtheorem{property}{Property}[section]

% Custom operations

\newcommandx{\idx}[2][2=n]{\ensuremath{\left(#1\right)_{#2}} }

% \newcommandx{\innerp}[2][1=\_, 2=\_]{\ensuremath{\langle#1, #2\rangle} }
% \newcommandx{\taskop}[2][1=\_, 2=\_]{\ensuremath{(#1 | #2)} }

% \newcommandx{\W}[2][1=kn, 2=m]{\ensuremath{W_{#2}^{#1}} }
% \newcommandx{\Wval}[2][1=kn, 2=m]{\ensuremath{e^{-\frac{2 \cdot j \cdot \pi \cdot #1 }{#2}}} }
% \newcommandx{\Wvalcoef}[2][1=2kn, 2=m]{\ensuremath{e^{-\frac{#1 \cdot j \cdot \pi }{#2}}} }
% \newcommandx{\eulersvalue}[2][1=x, 2=+]{\ensuremath{\cos(#1) #2 j\sin(#1)} }

% \DeclarePairedDelimiter\abs{\lvert}{\rvert}


% Title

\title{Digital Signal Processing. Final}
\author{Tymur Lysenko}
\affil{BS18-02SE, Innopolis University}
\date{Date of Birth: 05.07.1999}


% Document

\begin{document}

\maketitle

\section*{Variant}

\begin{flalign*}
  &day = 5& \\
  &month = 7& \\
  &year = 1999& \\
\end{flalign*}

\section{Task 1}

\subsection*{Problem statement}

Design (according to your actual variant) a filter to process signals with a period of length $m = year = 1999$ that passes (without any change) frequencies $\frac{2 k \pi}{m} = \frac{2 k \pi}{1999}$ for all $k \in [0 \dotsc (m - 1)]$ (namely, $k \in [0 \dotsc 1998]$) but $k = day = 5$ and $k = month = 7$. Explain the design algorithms and all design steps (providing references to the properties justifying the steps)!

\subsection*{Solution}

\subsubsection*{Plan}

In order to cancel out frequencies $\frac{2 \cdot 5 \cdot \pi}{1999} = \frac{10 \cdot \pi}{1999}$ and $\frac{2 \cdot 7 \cdot \pi}{1999} = \frac{14 \cdot \pi}{1999}$ 2 \textit{band-stop} filters must be designed (for both frequencies) for these frequencies, summed together (because of the linearity of DTFT) and applied to the signal.

\subsubsection*{Filter design to cancel frequency $\frac{10 \cdot \pi}{1999}$}

% TODO: Looks like DFT needs to be used instead of DTFT
% \begin{equation*}
%   X(e^{j\omega}) = \begin{cases}
%                      0, \, \omega = \pm \frac{14 \cdot \pi}{1999} \\
%                      1, \, \omega \neq \pm \frac{14 \cdot \pi}{1999}
%                    \end{cases}
% \end{equation*}

\subsubsection*{Filter design to cancel frequency $\frac{14 \cdot \pi}{1999}$}


\subsection*{Conclusion}

% TODO

\clearpage % TODO: remove
\section{Task 2}

\subsection*{Problem statement}

Design (according to your actual variant) an ideal low-pass filter to process infinite discrete signals that passes (without any change) only frequencies in range $\left[ -\frac{day}{year}, + \frac{month}{year} \right] = \left[ -\frac{5}{1999}, + \frac{7}{1999} \right]$. Explain the design algorithms and all design steps (providing references to the properties justifying the steps)!

\subsection*{Solution}

% TODO

\subsection*{Conclusion}

% TODO

\end{document}

\documentclass[a4paper]{article}

\usepackage[utf8]{inputenc}
\usepackage[margin=2cm]{geometry}
\usepackage[hidelinks]{hyperref}
\usepackage{textcomp,fullpage,enumitem,amssymb,amsmath,xcolor,cancel,gensymb,graphicx,indentfirst,xargs,mathtools}
\usepackage[T1]{fontenc}
\usepackage[affil-it]{authblk}
\usepackage[english]{babel}
\usepackage[amsmath]{ntheorem}

%\begin{align*}…\end{align*} if you want to fit an equation. 
%FOR PICTURES: include graphicsx 
%\includegraphics[scale=x]{name}
%double space and write caption in the center class

% Document settings

\setlength{\parskip}{1em}

% Theorems

\theoremstyle{break}
\theorembodyfont{\normalfont}
\newtheorem{definition}{Definition}[section]

\theoremstyle{break}
\theoremheaderfont{\itshape\hspace{-\theoremindent}}
\theoremindent10pt
\newtheorem{property}{Property}[section]

% Custom operations

\newcommandx{\innerp}[2][1=\_, 2=\_]{\ensuremath{\langle#1, #2\rangle} }
\newcommandx{\taskop}[2][1=\_, 2=\_]{\ensuremath{(#1 | #2)} }

\newcommandx{\W}[2][1=kn, 2=m]{\ensuremath{W_{#2}^{#1}} }
\newcommandx{\Wval}[2][1=kn, 2=m]{\ensuremath{e^{-\frac{2 \cdot j \cdot \pi \cdot #1 }{#2}}} }
\newcommandx{\Wvalcoef}[2][1=2kn, 2=m]{\ensuremath{e^{-\frac{#1 \cdot j \cdot \pi }{#2}}} }
\newcommandx{\eulersvalue}[2][1=x, 2=+]{\ensuremath{\cos(#1) #2 j\sin(#1)} }

\DeclarePairedDelimiter\abs{\lvert}{\rvert}


% Title

\title{Digital Signal Processing. Midterm}
\author{Tymur Lysenko}
\affil{BS18-02SE, Innopolis University}
\date{Date of Birth: 05.07.1999}


% Document

\begin{document}

\maketitle

\section{Task 1}

\subsection*{Problem statement}

Let us define the following binary operation \taskop in $C^{2}$: for any two vectors $v = (v_1, v_2)$ and $w = (w_1, w_2)$ let $\taskop[v][w] = (-1)^{day} \cdot month \cdot v_1w_1^* + year \cdot v_2^*w_2$. Is this operation a dot-product? If yes then prove, otherwise explain what dot-product axioms do hold and what do fail.

\subsection*{Solution}

First, substitute the values to the equation and get the unparameterized (with respect to the task parameters) formula for the operation:

\begin{itemize}
  \item $day = 5$
  \item $month = 7$
  \item $year = 1999$
\end{itemize}

\begin{equation*}
  \begin{split}
    \taskop[v][w]
      &= (-1)^{day} \cdot month \cdot v_1w_1^* + year \cdot v_2^*w_2\\
      &= (-1)^{5} \cdot 7 \cdot v_1w_1^* + 1999 \cdot v_2^*w_2\\
      &= -7 \cdot v_1w_1^* + 1999 \cdot v_2^*w_2
  \end{split}
\end{equation*}


Let us start with the definition of a \textbf{inner product} (alternatively, \textbf{dot product}).

\begin{definition}[Inner product]
  For a vector space $V$ over the field $F$, inner product is a mapping $\innerp: V \times V \rightarrow F$, s. t. the following 3 properties are satisfied for all vectors $u, v, w \in V$ and all scalars $a, b \in F$:

  \begin{property}[Conjugate symmetry] \label{prop:conj_sym}
    $$\innerp[u][v] = \innerp[v][u]^*$$
  \end{property}

  \begin{property}[Linearity in the first argument] \label{prop:liearity}
    $$\innerp[au + bv][w] = a\innerp[u][w] + b\innerp[v][w]$$
  \end{property}

  \begin{property}[Positive-definiteness] \label{prop:pos-def}
    $$
    \left\{
      \begin{array}{ll}
        \innerp[v][v] > 0, & v \neq 0 \\
        \innerp[0][0] = 0, & 
      \end{array}
    \right.
    $$
  \end{property}

\end{definition}

Assume that operation \taskop is an inner product. Then the properties above hold. Let's check this.

\subsection{Conjugate symmetry}

\begin{equation} \label{eqn:task1_source_eq}
  \taskop[v][w] = -7 \cdot v_1w_1^* + 1999 \cdot v_2^*w_2
\end{equation}

\begin{equation} \label{eqn:task1_eq_conjugation}
  \begin{split}
    \taskop[w][v]^*
      &= (-7 \cdot w_1v_1^* + 1999 \cdot w_2^*v_2)^* \\
      &= -7 \cdot (w_1v_1^*)^* + 1999 \cdot (w_2^*v_2)^* \\
      &= -7 \cdot w_1^*v_1 + 1999 \cdot w_2v_2^* \\
      &= -7 \cdot v_1w_1^* + 1999 \cdot v_2^*w_2 \\
  \end{split}
\end{equation}

It is clear from above that $\eqref{eqn:task1_source_eq} = \eqref{eqn:task1_eq_conjugation} = -7 \cdot v_1w_1^* + 1999 \cdot v_2^*w_2$, hence $\taskop[v][w] = \taskop[w][v]^*$, so the \textbf{conjugate symmetry \ref{prop:conj_sym}} property holds.

\subsection{Linearity in the first argument}

\begin{equation} \label{eqn:task1_eq_linearity}
  \begin{split}
    \taskop[av + bu][w]
      &= -7 \cdot (av_1 + bu_1)w_1^* + 1999 \cdot (av_2 + bu_2)^*w_2 \\
      &= -7 \cdot av_1w_1^* - 7 \cdot bu_1w_1^* + 1999 \cdot a^*v_2^*w_2 + 1999 \cdot b^*u_2^*w_2 \\
      &= -7 \cdot av_1w_1^* + 1999 \cdot a^*v_2^*w_2 - 7 \cdot bu_1w_1^* + 1999 \cdot b^*u_2^*w_2 \\
      &= -7 \cdot av_1w_1^* + 1999 \cdot (av_2)^*w_2 - 7 \cdot bu_1w_1^* + 1999 \cdot (bu_2)^*w_2 \\
      &= \taskop[av][w] + \taskop[bu][w]
  \end{split}
\end{equation}

As can be seen, the \textbf{linearity property \ref{prop:liearity}} does not hold for any field $F$. When $F = \mathbb{C}$ there exist $x \in \mathbb{C}$, s. t. $x \neq x^*$, so the operation \taskop is not an inner product of vector space $V$ over the field $\mathbb{C}$.

However, if $F = \mathbb{R}$, then $\forall x \in \mathbb{R}. x = x^*$, then \eqref{eqn:task1_eq_linearity} can be extended further:

\begin{equation*}
  \begin{split}
    \taskop[av + bu][w] &= \dotsc \\
      &= -7 \cdot av_1w_1^* + 1999 \cdot (av_2)^*w_2 - 7 \cdot bu_1w_1^* + 1999 \cdot (bu_2)^*w_2 \\
      &= -7 \cdot av_1w_1 + 1999 \cdot av_2w_2 - 7 \cdot bu_1w_1 + 1999 \cdot bu_2w_2 \\
      &= a(-7 \cdot v_1w_1 + 1999 \cdot v_2w_2) + b(-7 \cdot u_1w_1 + 1999 \cdot u_2w_2) \\
      &= a\taskop[v][w] + b\taskop[u][w]
  \end{split}
\end{equation*}

Hence, the \textbf{linearity in the first argument \ref{prop:liearity}} holds for \taskop when $F = \mathbb{R}$.

\subsection{Positive-definiteness}

Assume $v = 0$, then:

\begin{equation*}
  \begin{split}
    \taskop[0][0] = -7 \cdot 0 \cdot 0 + 1999 \cdot 0 \cdot 0 = 0
  \end{split}
\end{equation*}

Assume $v \neq 0$, then:

\begin{equation} \label{eqn:task1_eq_self}
  \begin{split}
    \taskop[v][v]
      &= -7 \cdot v_1v_1^* + 1999 \cdot v_2^*v_2 \\
      &= -7 \cdot \abs{v_1}^2 + 1999 \cdot \abs{v_2}^2
  \end{split}
\end{equation}

Let us find such combination of $v_1$ and $v_2$ that \eqref{eqn:task1_eq_self} is less than 0.

\begin{equation*}
  \begin{split}
    -7 \cdot \abs{v_1}^2 + 1999 \cdot \abs{v_2}^2 &< 0 \\
    1999 \cdot \abs{v_2}^2 &< 7 \cdot \abs{v_1}^2 \\
    \frac{1999}{7} \cdot \abs{v_2}^2 &< \abs{v_1}^2
  \end{split}
\end{equation*}

The last equation has infinitely many solutions both in $\mathbb{C}$ and $\mathbb{R}$. It means that there exists such vector $v \in V$ that $\taskop[v][v] < 0$, so the property does not hold and hence, the operation \taskop is not an inner product of vector space over fields $\mathbb{C}$ and $\mathbb{R}$.

\section{Task 2}

\subsection*{Problem statement}

Let us consider ($year$)-dimentional (complex) impulse-domain. Firstly, starting with/from the definition of the DFT, compute and give an explicit (in terms of trigonometric functions) representation for DFT of the following two signals:

\begin{itemize}
  \item $\delta_{(n - month)\ mod\ year}$
  \item $day^n$
\end{itemize}

Then starting with/from the definition of the IDFT validate that IDFT applied to the results returns exactly these signals.

\subsection*{Solution}

\subsection{Common definitions}

Assume that indexes start with 0.

Substituting task parameters, we get the following signals:

\begin{itemize}
  \item $\delta_{(n - 7)\ mod\ 1999}$
  \item $5^n$
\end{itemize}

Let us define $\W = \Wval$.

For $n$-th component of a periodic signal $x$ with period $m$ ($0 \leq n \leq m$) DFT is defined as follows:

\begin{equation} \label{eqn:dft}
  \begin{split}
    X_n = \sum_{k = 0}^{k = m - 1} x_k \W
  \end{split}
\end{equation}

For $n$-th component of a periodic signal $x$ with period $m$ ($0 \leq n \leq m$) IDFT is defined as follows:

\begin{equation} \label{eqn:idft}
  \begin{split}
    x_n = \frac{1}{m} \sum_{k = 0}^{k = m - 1} X_k \W[-kn]
  \end{split}
\end{equation}

\subsection{$\delta_{(n - 7)\ mod\ 1999}$}

\subsubsection{DFT}

We know, that $\delta_{(n - 7)\ mod\ 1999}$ is a periodic sequence of size 1999 with 1 at 7-th index and 0-es being values for other indexes. Using the definition \eqref{eqn:dft} we get that each element of the resulting sequence consists of a sum of zeroes plus $\W[7n][1999]$:

\begin{equation} \label{eqn:delta-dft}
  \begin{split}
    X_n &= \W[7n][1999] \\
        &= \Wval[7n][1999] \\
        &= \Wvalcoef[14n][1999] \\
        &= \eulersvalue[\frac{14n}{1999}\pi][-]
  \end{split}
\end{equation}

\subsubsection{IDFT}

Now apply the inverse transform \eqref{eqn:idft} to \eqref{eqn:delta-dft} to get back the original signal and validate that IDFT works:

\begin{equation*}
  \begin{split}
    x_n &= \frac{1}{1999} \sum_{k = 0}^{k = 1998} \W[7k][1999] \W[-kn][1999] \\
        &= \frac{1}{1999} \sum_{k = 0}^{k = 1998} \Wvalcoef[14k][1999] \Wval[(-kn)][1999] \\
        &= \frac{1}{1999} \sum_{k = 0}^{k = 1998} \Wvalcoef[(7 - n) \cdot 2k][1999]
  \end{split}
\end{equation*}


For $n = 7$ the series term is 1 and it is summed up 1999 times and then divided by 1999, which is exactly 1.

The term of the above series is the geometric progression with initial value 1 and common ratio $\Wvalcoef[(7 - n) \cdot 2][1999]$, so the sum is:

\begin{equation} \label{eqn:idft-series}
  \begin{split}
    x_n &= \frac{1}{1999} \cdot
           \frac{1 - \Wvalcoef[(7 - n) \cdot 2 \cdot 1999][1999]}{1 - \Wvalcoef[(7 - n) \cdot 2][1999]} \\
        &= \frac{1}{1999} \cdot
           \frac{1 - e^{-(7 - n) \cdot 2 \cdot j \pi}}{1 - \Wvalcoef[(7 - n) \cdot 2][1999]}
  \end{split}
\end{equation}

Apply the formula $e^{jx} = \eulersvalue$ to \eqref{eqn:idft-series}:

\begin{equation*}
  \begin{split}
    x_n &= \frac{1}{1999} \cdot
           \frac{1 - (\eulersvalue[(7 - n) \cdot 2 \pi][-])}{1 - \Wvalcoef[(7 - n) \cdot 2][1999]}
  \end{split}
\end{equation*}

For $n \neq 7$ the arguments of the expression form the nominator $\eulersvalue[(7 - n) \cdot 2 \pi][-]$ is multiple of $2\pi$. As it is known $cos$ of such arguments is always 1 and $sin$ is 0. Hence, the nominator equals to 0.


\subsection{$5^n$}

\subsubsection{DFT}

\begin{equation*}
  \begin{split}
    X_n &= \sum_{k = 0}^{1998} 5^k \cdot \W[kn][1999] \\
        &= \sum_{k = 0}^{1998} 5^k \cdot \Wval[kn][1999]
  \end{split}
\end{equation*}

The above series is a sum of geometric progression with initial value 1 and common ratio $5 \cdot \Wval[n][1999]$, then it can be simplified:

\begin{equation*}
  \begin{split}
    X_n &= \frac{1 - 5^{1999} \cdot \Wval[1999n][1999]}
                {1 - 5 \cdot \Wval[n][1999]} \\
        &= \frac{1 - 5^{1999} \cdot e^{-2 n \pi j}}
                {1 - 5 \cdot \Wval[n][1999]} \\
        &= \frac{1 - 5^{1999} \cdot (\eulersvalue[-2 n \pi][-])}
                {1 - 5 \cdot \Wval[n][1999]}
  \end{split}
\end{equation*}

At first, let us simplify the numerator $1 - 5^{1999} \cdot (\eulersvalue[-2 n \pi][-])$. The $\cos$ and $\sin$ have arguments that are multiples of $2\pi$ ($n \in \mathbb{N}$). It means that they equal  1 and 0 respectively, hence:

\begin{equation*}
  \begin{split}
    X_n &= \frac{1 - 5^{1999}}
                {1 - 5 \cdot \Wval[n][1999]} \\
        &= \frac{1 - 5^{1999}}
                {1 - 5 \cdot \eulersvalue[\frac{2}{1999} n \pi][-]}
  \end{split}
\end{equation*}

% \section{Task 3}

% \subsection{Problem statement}

\end{document}
